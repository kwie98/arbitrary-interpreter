%---configure pagelayout
\KOMAoptions{ % read documentation for KOMAScript providing the documentclass scrartcl
DIV=11,
BCOR=0mm,
paper=a4,
fontsize=11pt,
parskip=half,
twoside=false,
titlepage=true
}
\usepackage{pdfpages}
\usepackage[ %Set linespacing
singlespacing %onehalfspacing,doublespacing
]{setspace} 

\usepackage{pgffor, ifthen,tikz}
\usepackage{lastpage}
\usepackage[headsepline,automark,komastyle,nouppercase]{scrpage2} %Configure headline and footer

\clearscrheadings
\setlength{\headheight}{2.5\baselineskip}
\ihead[]{\authA \, (\matA) \\ \authB \, (\matB)} 
\ohead{Gruppe \grpnr}
\cfoot[]{\thepage\ von \pageref*{LastPage}}
\usepackage[left=2cm,right=2cm,top=1.5cm,bottom=1cm,includeheadfoot]{geometry}


%better positioning of floatings
\usepackage{float}
\renewcommand{\floatpagefraction}{.75} % standard: .5
\renewcommand{\textfraction}{.1} % standard: .2
\renewcommand{\topfraction}{.8} % standard: .7
\renewcommand{\bottomfraction}{.5} % standard: .3
\setcounter{topnumber}{3} % standard: 2
\setcounter{bottomnumber}{2} % standard: 1
\setcounter{totalnumber}{5} % standard: 3


%---Language and umlauts
\usepackage[utf8]{inputenc} %Set UTF-8 encoding, enables ä,ö,ü etc.
\usepackage[ngerman]{babel} %Set document language to ngerman (new german)  
\usepackage[% improved hyphenation
expansion=true,
protrusion=true
]{microtype}


%---Mathmatics (AMS packages )
\usepackage{amsmath} %generell math enviorments e.g. align
\usepackage{amsfonts}
\usepackage{amssymb} 
\usepackage{amsthm} %math theorems
\usepackage{upgreek}%provide special form of greek letters e.g. \upmu


%---Units
\usepackage[decimalsymbol=comma,separate-uncertainty = true]{siunitx}


%---tables and imgaes
\usepackage{graphicx} %provide \includegraphics[options]{name}
\usepackage{epstopdf} %enable the use of eps-graphics
\usepackage[hypcap]{caption} %captions out of floating-enviorments(figure,table)
\usepackage{booktabs} %extra lines in tabulars
\usepackage{flafter} %Place floating-enviorments after references.
\usepackage[ %
section %latest ancor for floating-enviorments
]{placeins}


%---Hyperlinks
\usepackage{hyperref} %create table of content and references as links
\hypersetup{
%
%Colors 
colorlinks=true, 
breaklinks=true, 
citecolor=red, 
linkcolor=blue, 
menucolor=red, 
urlcolor=cyan,
%
%pdf-bookmarks
bookmarksopen=false, 
bookmarksopenlevel=0,
% 
%pdf-data
% pdftitle={\titel}, 
% pdfauthor={\writer}, 
% pdfcreator={\writer}, 
% pdfsubject={\titel}, 
% pdfkeywords={\titel} 
%
%misc
plainpages=false,% zur korrekten Erstellung der Bookmarks 
hypertexnames=false,% zur korrekten Erstellung der Bookmarks 
% hyperindex=true,
}
\newcommand{\notes}[3][\empty]{%
	\foreach \n in {1,...,#2}{%
		\ifthenelse{\equal{#1}{\empty}}
		{\rule{#3}{0.5pt}\\}
		{\rule{#3}{0.5pt}\vspace{#1}\\}
	}
}


%Abbildungsname umbenennen ("Abb." statt "Abbildung")
\renewcaptionname{ngerman}{\figurename}{Abb.}


%Schaltkreise
\usepackage[]{circuitikz}
\usepackage{tikz}


%Subsubsubsections
\newcommand{\subsubsubsection}[1]{\paragraph{#1}\mbox{}\\}
\setcounter{secnumdepth}{4}
\setcounter{tocdepth}{4}


% schöne eingekreiste Zahlen
\newcommand{\kreis}[1]{\,\unitlength1ex\begin{picture}(2.5,2.5)%
\put(0.75,0.75){\circle{2.5}}\put(0.75,0.75){\makebox(0,0){#1}}\end{picture}}
% der Befehl ist dann \kreis{3} (die unschöne Variante ist \textcircled{1})


%in Formeln durchstreichen (kürzen)
\usepackage{cancel}